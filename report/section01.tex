\section{Introduction to Well Logging Services}
Well logging is a critical formation evaluation technique employed to characterize subsurface geological formations and fluid content within a borehole. The process involves deploying sophisticated sensor suites into the well—typically via wireline conveyance or logging-while-drilling (LWD) assemblies—to acquire continuous, depth-correlated measurements of physical rock properties. These acquisitions generate data records, known as logs, which facilitate the detailed analysis of the reservoir without the prohibitive cost and time associated with extensive physical coring. This diagnostic process is versatile, applicable in both “open-hole” environments to assess potential pay zones immediately after drilling, and “cased-hole” environments to evaluate cement integrity and monitor saturation changes during the well’s production life.

To construct a comprehensive petrophysical model, a diverse array of logging tools is utilized to quantify specific parameters. Fundamental services include Gamma Ray logs for lithological identification and shale volume estimation, alongside Resistivity logs, which are essential for discriminating between non-conductive hydrocarbons and conductive formation water. To assess reservoir storage capacity, porosity logs—comprising density, neutron, and sonic measurements—are utilized to quantify void space volume.

\begin{figure}
	\centering
	\includegraphics[width=20em]{images/well_log.png}
	\caption{Well Logging Process}
	\label{fig:well_log_example}
\end{figure}

Complementing these are auxiliary tools such as caliper logs, which map borehole geometry to assess stability, and formation testers that isolate specific zones to capture pressure gradients and in-situ fluid samples, validating the physical state of the reservoir.

The integration of this data is indispensable for accurate reservoir characterization, reserve estimation, and field development planning. By accurately delineating rock types, fluid contacts, and permeability, operators can mitigate drilling risks, determine the commercial viability of a zone, and optimize completion strategies, such as precise perforation placement. Furthermore, cased-hole logs provide essential assurance regarding hydraulic isolation and zonal integrity. Ultimately, well logging serves as the foundation for robust reservoir modeling, enabling data-driven decisions that maximize economic efficiency and hydrocarbon recovery throughout the asset’s entire lifecycle.
